\chapter{Ensemble di classificatori}\label{ensemble-di-classificatori}

L'utilizzo di più classificatori per creare un ensemble (insieme) è frequentemente utilizzato nel Deep Learning, si basa sull’utilizzo di più
reti, o modelli, per risolvere un problema di classificazione. L’idea di base è che utilizzando più punti di vista diversi, uniti secondo una regola, si possano raggiungere accuratezze
più alte ed errori meno frequenti.

Ogni singolo classificatore fornisce in output la propria decisione che consiste nella classe cui
ha assegnato il pattern.\cite{unibo_maltoni_ml}\cite{zhi_enseble}. 

\section{Majority Rule}\label{majority-rule}

Il più semplice modo per combinare più classificatori è a livello
di decisione, con questa metodologia, ogni classificatore fornisce in output la classe alla quale ha segnato il Pattern.  Con \(n\) classificatori: ognuno vota una classe per il dato
pattern, e semplicemente  assegno il Pattern (cioè classifico il Pattern) come la classe maggiormente votata.

Più formalmente.
Se \( \left\{C_{1}, C_{2}, \ldots C_{n c}\right\} \) è un insieme di \( n c \) classificatori, \( e \)
\[
\theta_{i j}=\left\{\begin{array}{cc}
1 & \text { se } i \text { è la classe votata da } C_{j} \\
0 & \text { altrimenti }
\end{array} \quad(1 \leq i \leq s, 1 \leq j \leq n c)\right.
\]
Allora il pattern è assegnato alla classe \( t \) tale che:
\[
t=\operatornamewithlimits{argmax} _{i=1_{\ldots} s}\left\{\sum_{j=1 \ldots n c} \theta_{i j}\right\}
\]