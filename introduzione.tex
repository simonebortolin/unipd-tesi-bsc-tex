\chapter{Introduzione}\label{introduzione}

La laringoscopia è una procedura medica utilizzata per ispezionare la laringe e per diagnosticare, ed eventualmente curare, i disturbi della laringe e delle corde vocali.

In particolare ci sono due tipi di laringoscopia, quella indiretta e quella diretta. La prima fa uso di un semplice specchio e di una fonte luminosa, la seconda richiede l'uso di un laringoscopio e una anestesia.

Il laringoscopo è uno strumenti che grazie a una fibra ottica, una sorgente luminosa e una telecamera permette di osservare nei minimi dettagli la laringe e tutti gli elementi che la costituiscono come le epiglottide e le corde vocali\cite{giorgio_cenni_2008}. In particolare permette di aiutare a diagnosticare malattie come la laryngeal squamous cell carcinoma (SCC), la cui diagnosi precoce riduce la mortalità del paziente\cite{moccia_larynge}.

L'obiettivo di questa tesi è quello di usare l'uso di algoritmi di apprendimento automatico basati sul transfer learning e sul  preprocessing di immagini  per la classificazione dei frame di una laringoscopia, in particolare la selezione degli informative-frame e lo scarto di tutti i frame non utili come frame pieni di saliva o non a fuoco.

