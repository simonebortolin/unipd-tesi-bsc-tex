\chapter{Conclusioni}\label{conclusioni}

Dai risultati in \cref{tab:specchio-single} si  nota come un semplice doppio allenamento 2R migliora le performance generali della rete, in particolare si arriva a un 97.5\% di performance generali con un data augmentation basico. 

Con l'inserimento di rumore le performance aumentano in 1R e peggiorano in 2R, con l'aggiunta di un semplice filtro per il contrasto hardcoded le performance raggiungono lo stesso risultato indipendente se l'allenamento è 1R o 2R.

Nonostante ciò con l'inserimento di DTC e rumore le performance calano, lasciando intuire che la rete in generale prende in considerazioni alcune feature secondarie, che con la soppressione casuale di frequenze e il rumore bianco spariscono.

Un importante risultato è comunque verificare che l'allenamento 2R incrementi in generale le performance. Inoltre un dato molto importante è  quello che gli informative frame vengono quasi sempre catalogati correttamente nel test set.