%definisco il layout dell'abstract
\def\changemargin#1#2{\list{}{\rightmargin#2\leftmargin#1}\item[]}
\let\endchangemargin=\endlist 

%Genero l'ambiente per l'abstract
\newcommand\summaryname{Abstract}
\newenvironment{Abstract}%
    {\begin{center}%
    \bfseries{\summaryname} \end{center}}
    
\begin{Abstract}
\begin{changemargin}{1cm}{1cm}
In questa tesi si introducono alcuni metodi di preprocessing di immagini e data augmentation per creare nuove ensemble a partire da quelle esistenti in combinazione con un metodo di addesstramento di tipo two round tuning al fine di massimizzare le prestazioni di una \gls{cnn}.

In particolare useremo il transfer learning riutilizzando  una \gls{cnn} preaddestrata per effettuare una selezione degli  informative-frame in una laringoscopia. Una buona selezione degli informative-frame permette di ottimizzare le elaborazioni e la visione da parte dei medici e tecnici competenti.

Tutto questo viene fatto attraverso una rete \gls{cnn} di tipo deep preaddestrata con l'uso del trasnfer learning, data augmentation e preprocessing.

In particolare oltre ai classici metodi di rotazione e mirroring dell'immagine viene utilizzato pure una DCT e varie tecniche di preprocessing per dare in pasto alla rete una immagine più dove è maggiormente analizzabile.

\end{changemargin}
\end{Abstract}