%definisco il layout dell'abstract
\def\changemargin#1#2{\list{}{\rightmargin#2\leftmargin#1}\item[]}
\let\endchangemargin=\endlist 

%Genero l'ambiente per l'abstract
\newcommand\summaryname{Sommario}
\newenvironment{Abstract}%
    {\begin{center}%
    \bfseries{\summaryname} \end{center}}
    
\begin{Abstract}
\begin{changemargin}{1cm}{1cm}
Nel presente documento si discutono e validano alcune tecniche per l'aumento delle prestazioni di una rete neurale convoluzionale. Verranno prese in considerazione metodologie di preelaborazione, aumento dei dati e trasferimento delle conoscenze. 
In particolare, analizziamo l'utilizzo del Transfer Learning unito al "Two Round Tuning" per migliorare le capacità riconoscitive della rete. 

Nel campo medico, sempre più spesso si ricercano metodi per migliorare i set di dati presenti ed ottenere le miglior prestazioni con quelli già presenti. Studiando fotogrammi endoscopici laringei, siamo riusciti a determinare quali combinazioni di tecniche presenti nella letteratura diano i migliori risultati nel determinare la presenza o meno di tumore, facilitando la diagnosi medica.
\end{changemargin}
\end{Abstract}