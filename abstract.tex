%definisco il layout dell'abstract
\def\changemargin#1#2{\list{}{\rightmargin#2\leftmargin#1}\item[]}
\let\endchangemargin=\endlist 

%Genero l'ambiente per l'abstract
\newcommand\summaryname{Sommario}
\newenvironment{Abstract}%
    {\begin{center}%
    \bfseries{\summaryname} \end{center}}
    
\begin{Abstract}
\begin{changemargin}{1cm}{1cm}
In questa tesi si introducono e si validano alcuni metodi per il preprocessing di immagini, il data augmentation per creare nuove ensemble a partire da quelle esistenti. In combinazione con un metodo di addestramento di tipo two round tuning al fine di massimizzare la fase di allenamento di una rete neurale.

In particolare useremo il transfer learning riutilizzando una rete pre addestrata per effettuare una selezione degli informative-frame in una laringoscopia. Una buona selezione degli informative-frame permette di ottimizzare le elaborazioni e la visione da parte dei medici e tecnici competenti.

Tutto questo viene fatto attraverso una rete deep pre addestrata con l'uso del trasnfer learning, data augmentation e preprocessing.

In particolare oltre ai classici metodi di rotazione e mirroring dell'immagine viene utilizzata pure una DCT e varie tecniche di preprocessing per allenare la rete con un numero superiori di immagini a partire da un dataset di dimensioni contenute.

\end{changemargin}
\end{Abstract}